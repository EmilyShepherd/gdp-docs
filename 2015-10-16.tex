\documentclass{article}
\usepackage[english]{babel}
\usepackage{minutes}
\usepackage[letterpaper, margin=1in]{geometry}
\usepackage{mathtools}

\pagestyle{plain}

\setlength{\parindent}{0em}
\setlength{\parskip}{1em}

\begin{document}
\selectlanguage{english}

\begin{Minutes}{Initial Team Meeting}
\participant{\
    Emily Shepherd, \
    Mohit Gupta, \
    Dan Playle, \
    Toby Finch, \
    Calin Pasat,
    Alex Weddell,
    James Myres,
    Rohan Gaddh}
\minutesdate{16 October 2015}
\starttime{15:00}
\endtime{15:30}
\location{59/4231}
\maketitle

\topic{Introductions}
Dr Weddell started the call and began the meeting - he suggested that each member of the team introduce
themselves. Dan began, stating that he would take the lead on Machine Learning. Calin then stated that he
would be more involved in the hardware implementation side. Next, Emily introduced herself as the primary
for the embedded implementation. Toby then said that he would be focusing on the abstract design of the
algorithm. Finally, Mohit greeted the customer and informed him that he would also be managing the
hardware side of the project.

\topic{Project Plan}
Emily began proceedings by asking James if he had seen the proposed project brief - James said that
he had not, so Emily briefly went over it. James said that this sounded sensible and asked if the group
were aware what ARM's particular interests were - Emily said that the team would like clarification on
that so James explained that ARM were working on their subthreshold Cortex M0+ processor and briefly
mentioned that this would mean reduced voltages (of around 1.2V) and reduced frequencies. He stated
that any software, therefore, that was to be run on the device would need to be designed in the knowledge
that it would have very few processessing cycles availible to it and that the memory would be very small.
He mentioned that, as a result of this, the systems they had built so far had tended to swap between
collecting data and processing data, rather than processing data on the fly, as the processing ability
was so low. Emily clarified that this mean that ARM's interests were to see what was the most the team
could ``get away with'' running on the device, which James agreed with.

Dr Weddell asked James to clarify what memory would be availiable - James responded that the system
should ideally use between 8 and 12 kB, including code and data, if it is to feasibly be used on their
subthreshold device - he said that if it had to go higher, it could still be used on emulators but there
would be no way to test this on the real device. Emily stated, therefore, that the team would use 12kB
as a ``reasonably hard'' cut off point.

\topic{Testing Platform}
Dr Weddell asked if it would be sensible to work on the board itself, or resort to emulation. James
stated that a board may be able to made availiable to the group, and that there were some people in
Southampton with the required knowledge to work with it, but this was not essencial and it would
make more sense to emulate the platform during development. He said they would think about making a
board availiable if the algorithm worked well - Emily summerised this as a possibility but not a
requirement for the group's aims.

Dr Weddell then asked about the team's proposed plan of using an FPGA board, stating that he was not
sure if this was necissary. James responded that he was not sure but he suspected it was a sensible
way of proceeding. Dr Weddell mentioned the possibility of using an M1, Mohit stated this would not
be helpful for the group and mentioned that he had been looking at DesignStart. James agreed with
Mohit and suggested that the embed M0 would be a feasible possibility. Mohit said he had looked into
this, although the team were unsure on it as the data sheet listed a 40\% error.

Rohan suggested using both the Embed M0 and the FPGA - the team agreed with this. Dr Weddell asked if
James would be able to make an Embed M0 availiable to the team, to which he responded that one would
need to be purchased. Calin confirmed that the budget could handle this.

\task*{Order Embed M0}

\topic{Accelerometers}
Dr Weddell asked about the accelerometers and stated that he assumed these would not need to be under
the same power constraint. James confirmed that he was unaware of any low-power sensors.

\topic{Progress Seminiars}
Dr Weddell suggested the possibility of inviting James and Rohan to the progress Seminars. After some
discussion, it was decided that Rohan would not be able to make either of the dates, however James
may be free for the second progress seminar.

\topic{Ongoing Contact}
Dr Weddell asked if ongoing contact would be appropriate - and how the team could ask any questions if
required. James stated that it would probably be better to remain in contact with Rohan.

\decision*{Use Rohan as the point of contact}

\end{Minutes}
\end{document}