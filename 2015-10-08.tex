\documentclass{article}
\usepackage[english]{babel}
\usepackage{minutes}
\usepackage[letterpaper, margin=1in]{geometry}

\pagestyle{plain}

\setlength{\parindent}{0em}
\setlength{\parskip}{1em}

\begin{document}
\selectlanguage{english}

\begin{Minutes}{Supervisor Meeting 2}
\participant{\
    Emily Shepherd, \
    Mohit Gupta, \
    Dan Playle, \
    Toby Finch, \
    Calin Pasat, \
    Dr Geoff Merrett, \
    Dr Alex Weddell}
\minutesdate{8 October 2015}
\starttime{13:00}
\endtime{13:40}
\location{59/4227}
\maketitle

\topic{Project Brief}
Emily apologised for sending the latest draft of the brief so close to the meeting. Dr Merret said
he had not yet had a chance to go over it, so Emily went over the changes since the previous
version; she stated that his points had all been accepted by the team, and explained that many of
them they had been planning to do anyway, but agreed that the brief was too vague. She answered
Dr Merrett's question [Note: Via email before the meeting, Dr Merrett had queried what the Brief
meant when it referred to ``other Service Oriented Architectures"] stating that the team had meant
algorithms running on unconstrained systems.

Dr Merrett said he was happy with everything - his only note
was that the brief said ``an algorithm... for execution on... [a] subthreshold device". He suggested
adding ``suitable for" to make it clear that the team would not actually be working with the
subthreshold device in question. Emily pointed out that Dan had made a change which she had not yet
had a chance to see - Dan explained this was simply to add a reference to existing tools used for
Machine Learning, such as Waka. The group were all happy with this. Dr Weddell had no issues, other
than a typo he had noticed in the Gantt chart. Emily apologised and said she'd make sure to spell
check it before submitting.

Dr Merrett said that the brief should also be emailed to the client - Emily asked if he would like
to be CCed on that, to which Dr Merrett replied that he would. Toby asked if the Gantt charts should
also be sent over, but Dr Merrett said it would probably be better to do without those.

\decision*{Add ``suitable for'' into the brief}
\decision*{Accept Dan's change}
\task{Toby}{Email the brief to the custonmer}

\topic{Project Administration}
Mohit asked about gaining access to the code required for the M0 - Dr Merrett stated that he should
try signing up to ARM's site, although mentioned that there might be a single person who is signed
up for the whole University who he should contact.

Dr Merrett then asked how things had gone with the group so far for the week, and Toby responded
that they had been working on the Ethics Approval. Emily mentioned that the plan is to complete
the required forms this week, ready for the possible two weeks of processing time. Dr Merrett
said that, in order to submit the request, the team would have to have a clear idea of what their
testing would be - Toby said he had planned to write about the excersises, Emily added that testing
non-excersise motions would be sensible too, to ensure the algorithm does not give false-positives.

Dr Weddell asked to be reminded if the algorithm was aimed at inflight usage, and stated that that
would add some convienient constraints on the requirements, as measuring activities such as walking
up stairs may not be required. Dr Merrett pointed out that some planes do in fact have stairs,
however the team accepted Dr Weddell's point - Emily added that movement due to turbalence may
need to be considered.

Dr Merrett also pointed out that there may be two phases of testing required: the first to obtain
the data, and a second to verify that the system worked correctly. Emily said they had planned this
but weren't sure if this should be submitted as separate requests - Dr Merrett stated that one
request would be appropriate and informed the group that, while the ethics committee do require
precision and detail in the application, if possible be as general as possible within those
constraints.

Emily briefly touched upon to topic of risk assessments - Dr Merrett suggested that allergies may
be one as the device will involve skin contact. He suggested that using the device over clothes
would be a sufficiant way to avoid this risk.

Mohit then reminded the group about Laboratory Risk assessments. Emily handed their completed
risk statements to Dr Merrett who counter signed each of them. He asked why he only needed to sign
one per team member, as other teams had given him more to sign. Emily stated that she believed
copying the forms once signed would be fine which Dr Merrett accepted.

\topic{Hardware}

Emily brought up a topic of confusion for the group at their previous meeting\footnote
{2015-05-05, Background Research, Paragraph 2}: the M0's long term memory capability.
Dr Merrett was unsure what flash memory would be availiable and this may be something worth
talking to the customer about. Dr Weddell looked up the M0 and found there were two versions:
one with 256B of flash memory and 32kB of RAM, and a smaller unit with a total of 32kB flash
and RAM, of which 8kB is RAM.

\task{Toby}{Ask the customer about long term memory}

\topic{Project Plan}
Dr Merrett asked what the team's project plan was.

\subtopic{Hardware}
Emily started with hardware - she went over the contents of the Gantt chart, stating that they
planned to spend the next week begining the process of acquiring the required items, and they
had planned to have a working system built just in time for the planned data collection week
if the suppliers are slow.

Dr Merrett asked if this plan implied the team were planning on using the
device itself to collect the training data, which Emily confirmed. He suggested this was a
poor strategy as it would be easier to simply use a fully working device to collect high
precision data. Mohit stated that the formats may be different which could cause a problem,
however Emily interjected that she believed that data conversion shouldn't be difficult,
assuming both formats are adequately documented.

\subtopic{Software}
Emily then explained the software side of the project's planning, stating that the plan is
to spend the next couple of weeks dealing with ``administration" tasks such as the ethics
approval, quickly leading onto the reasearch phase. She said the team had decided to research
Machine Learning for recognising excersises on non constrained systems at first, followed by
research into Machine Learning in general on constrained systems.

She then said the next few weeks were dedicated to starting to develop an algorithm, first
just in a high level language such as Java, both to help with planing the algorithm and to
give the hardware side time to get the system working before work started. Finally, she said
they would move into the embedded development of the team's chosen algorithm.

Dr Merrett started that this course of action was sensible and asked if the research was going
to focus on practical solutions or academic research. Emily responded that they had planned
both, although she didn't have a strong response as they had not made a start on the literature
at that point.

\topic{Relavence of the FPGA}
At this point Dr Weddell asked Dr Merrett why the original specification had specified an
FPGA. Dr Merrett responded that this was to clock it slower, however Dr Weddell was not
convinced this could not be achieved without the FPGA board. Dr Merrett said he could not
remember the details, but he believed it to be more complex. He suggested asking the customer
about it.

\topic{Algorithm Idea}
Dr Merrett asked if the team had any initial ideas about their approach to the algorithm. Emily
responded that, due to her lack of experience with Machine Learning and she was not in a good
position to make authoritative opening statements. Toby mentioned that Dan had been keen at
Machine Learning. Dan stated that he had had a preliminary look into what was availiable,
specifically Waka which he said he hadn't used before. He also said he had found a solution
but that it was in Java.

Dr Merrett said this was a good start, and gave some advice about Waka: he stated that it was
a very useful and powerful tool, but it should not be used exclusively. That is, the team should
not simply plug data into it and blindly accept the first thing it outputs.

\topic{Immediate Plan}
Dr Merrett asked for a more detailed look at the plan for the next two weeks. Emily stated that
it was largely the administration tasks for a bit, but then the team would get stuck into
research while the hardware side started on the development. Dr Merrett said this was good,
although given the five person team, he would like to see at least one person doing something
more practical early on, so that the team would have something solid to display at the first
progress seminar.

Dr Merrett then clarified that the team is classed as an ``EE" team for the purposes of the
seminars, which means the later time slots would be used - Emily thanked him and stated that
they had planned the earlier ones in the Gantt charts for safety but she would move them
prior to submitting the brief. Dr Merrett then explained that the first progress seminar was
a more general one, so there may not be any need to have too much done for it. For example,
he stressed the importance of picking the best algorithm, rather than simply going for the
first one the team found.

\topic{Team Roles}
Dr Merrett asked about team roles. Emily listed what these were\footnote
{These were decided in the group meeting of 2015-09-30}; she stated that she had taken the
position of leader and minute taker, Toby was the Company Representative, and Calin was the
budget holder. Calin reminded the group that he was also in charge of acquiring the participants
for the data study. Dr Merrett asked if this might make them feel obligated to take part, but
Emily stated that they would have a formal consent form.

Emily then explained their roles for the project work: she started with Calin and Mohit, who
were in charge of developing the hardware. She then said that she and Toby were taking the
lead for the research, that she, Toby and Dan were to be the main workers on coming up with
the algorithm, and that she and Dan would be the primaries for the embedded programming work.

Dr Merrett queried if this would mean the others would not contribute to other areas - Emily
clarified that that was not at all the case; all members would contribute to everything and
the roles she had just listed were mearly the people who would head up each section of work.

\topic{Next Meeting}
Dr Merrett reminded the group that he would not be present next week. Emily apologised and
stated that she would not be able to make the standard meeting slot next week due to a
doctor's appointment. After some discussion, the group settled on 11:30 on Tuesday in
Dr Weddell's office.

Dr Merrett asked for a reminder of the standard meeting time for each week.

\decision*{Next Meeting: 11:30, Tuesday 13 October 2015}
\task{Emily}{Remind Dr Merrett of meetings}

\end{Minutes}
\end{document}