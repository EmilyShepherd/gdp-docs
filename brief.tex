\documentclass{article}
\usepackage[letterpaper, margin=1in]{geometry}
\usepackage{pgfgantt}
\usepackage{pdflscape}

%% DRAFT %%
	\usepackage[printwatermark]{xwatermark}
	\usepackage{xcolor}
	\usepackage{graphicx}
	\usepackage{tikz}

	\newsavebox\mybox
	\savebox\mybox{\tikz[color=red,opacity=0.2]\node{DRAFT};}
	\newwatermark*[
	  allpages,
	  angle=45,
	  scale=10,
	  xpos=-30,
	  ypos=15
	]{\usebox\mybox}
%% END DRAFT %%

\title{School of Electronics And Computer Science \\
ELEC6200 MEng Group Design Project \\
Project Specification And Plan}
\author{Group 2}
\date{October 2015}

\setlength{\parindent}{0em}
\setlength{\parskip}{1em}

\begin{document}

\maketitle

\textbf{Title:} Ultra-low-power exercise monitoring applications for sub-threshold micro-controllers

\textbf{Supervisors:} Dr Geoff Merrett and Dr Alex Weddell

\textbf{Team Members:}
\begin{itemize}
    \item Emily Shepherd (ams2g11)
    \item Mohit Gupta (mg8g12)
    \item Toby Finch (tlf1g12)
    \item Dan Playle (djap1g11)
    \item Calin Pasat (cp10g12)
\end{itemize}

\textbf{Customer:} James Myers, ARM Research, Cambridge (James.Myers@arm.com)

\textbf{Project Specification:}

This project will research and develop a system for monitoring exercises performed by a human wearer,
more specifically exercises performed to avoid suffering from deep vein thrombosis (DVT)
e.g. http://www.virgin-atlantic.com/gb/en/travel-information/your-health/inflight-exercise.html.
However, the additional constraint on this project is that the algorithms should be designed for
execution on an ultra-low-power (a few uW) subthreshold ARM Cortex-M0+ currently being investigated
at ARM Research. This requires that the algorithm can execute on a processor clocked at a few hundred
kHz and from limited memory.

The project has a number of subtasks that need to be achieved (some of which can be performed concurrently):
\begin{enumerate}
\item Create a test platform by obtaining and synthesise the ARM DesignStart Cortex-M0 to an FPGA,
and be able to clock it at frequencies from 100s of kHz to a few MHz, and a single memory (holding
program and data) of 32 kB (capable of being enabled in 4kB blocks)
\item Investigate existing algorithms and approaches for monitoring exercises/activities, and develop your own optimised algorithm(s) suitable for the target platform (note, you may have to consider an accuracy/power trade-off in doing this).
\item Implement to allow demonstration and evaluation of the chosen approach(es), including making estimates of the energy/power consumption compared to existing SoA approaches.
\end{enumerate}

You may also choose to use a COTS micro-controller development board for testing and development of the algorithm, if desired (and if manpower allows).

As potential extensions, you may choose to look at 1) alternative algorithms that could increase accuracy or reduce power (lower frequency and memory use), 2) use of the system/platform to execute algorithms for other exercises/activities etc. As it is likely to involve human experiments/data, you may need to obtain ethical approval for the project as soon as possible.

\newpage
\begin{landscape}

\begin{ganttchart}[
		hgrid,
		vgrid,
		x unit=4mm,
		time slot format=isodate
	]{2015-09-28}{2015-11-15}
	\gantttitlecalendar{month=shortname, week} \\

	\ganttgroup{Project Brief}{2015-09-30}{2015-10-09} \\
	\ganttbar{Agree Project Brief}{2015-09-30}{2015-10-08} \\
	\ganttlinkedmilestone{Project Brief Due}{2015-10-09} \ganttnewline[thick]

	\ganttgroup{Algorithm Work}{2015-10-05}{2015-11-01} \\
	\ganttbar{Background Research}{2015-10-05}{2015-10-25} \\
	\ganttbar{Algorithm Design}{2015-10-12}{2015-11-01} \ganttnewline[thick]

	\ganttgroup{Movement Data Study}{2015-10-05}{2015-10-25} \\
	\ganttbar{Search for Volunteers}{2015-10-12}{2015-10-21} \\
	\ganttbar{Apply for Ethics Approval}{2015-10-05}{2015-10-18} \\
	\ganttlinkedbar{Collect Movement Data}{2015-10-19}{2015-10-25} \ganttnewline[thick]

	\ganttgroup{Progree Seminar}{2015-10-19}{2015-10-23} \\
	\ganttbar{Seminar Preparation}{2015-10-19}{2015-10-22} \\
	\ganttmilestone{Progress Seminar 1}{2015-10-23}
\end{ganttchart}

\begin{ganttchart}[
		hgrid,
		vgrid,
		x unit=4mm,
		time slot format=isodate
	]{2015-11-16}{2016-01-03}
	\gantttitlecalendar{month=shortname, week=8} \\

	\ganttbar[
		inline,
		bar height=1,
		bar top shift=0,
		bar/.append style={fill=red, draw=red},
		bar label font=\color{yellow}
	]{Christmas Holidays}{2015-12-14}{2016-01-03} \\

	\ganttmilestone{Progress Seminar 2}{2015-11-20} \\
	\ganttlinkedbar{Changes}{2015-11-20}{2015-12-14} \\
	\ganttbar{Report Writing}{2015-12-14}{2016-01-03}

\end{ganttchart}

\begin{ganttchart}[
		hgrid,
		vgrid,
		x unit=4mm,
		time slot format=isodate
	]{2016-01-04}{2016-02-21}
	\gantttitlecalendar{month=shortname, week=16} \\

	\ganttmilestone{Report Due}{2016-01-28} \\
	\ganttmilestone{Reflection Report}{2016-02-01} \\
	\ganttmilestone{Poster and slides due}{2016-02-04} \\
	\ganttmilestone{Final Presentation}{2016-02-20}
\end{ganttchart}

\end{landscape}

\end{document}
